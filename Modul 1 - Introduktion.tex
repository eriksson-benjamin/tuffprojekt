\documentclass[9pt]{beamer}

\usepackage{beamerpreamble}
\usepackage[swedish]{babel}
\usepackage{minted}
\usepackage{comment}
\usemintedstyle{vs}
\usepackage{xcolor}
\usepackage{tikz}

\renewcommand{\ttdefault}{cmtt}

\newcommand*\mean[1]{\bar{#1}}

\title{Datalaboration - Förberedande tutorial}
\author[benjamin.eriksson@physics.uu.se]{Benjamin Eriksson  \\ \tiny{med inspiration från} \\ \scriptsize{M. Isacson, M. Ellert, M. Olvegård, och B. Lindgren}}
\institute[Uppsala universitet]{{\small Avdelningen för tillämpad kärnfysik \\ Institutionen för fysik och astronomi} \\ \uulogo}
\date{{\small Reviderad}\\ \today}

\begin{document}
    \frame{\titlepage}

    \begin{frame}{Innan du börjar}
    Se till att du har
        \begin{itemize}
            \item aktiverad version av Matlab på din dator \\ \url{http://www2.teknat.uu.se/student/matlab/instruction/InstallStudentMatlab.pdf} 
            \item laddat ner \textit{tutorial\_1.mlx}, \textit{tutorial\_2.mlx} och \textit{tutorial\_3.mlx} 
            \item länkar finns överst på sidan
        \end{itemize}
    \end{frame}
    
    \begin{frame}{Introduktion}
        Fem föreläsningsmoduler
        \begin{enumerate}
            \item Introduktion
            \item Mätvärdesbehandling
            \item Minstakvadratmetoden
            \item Fortplantning av osäkerheter
            \item Återkommande intervall
        \end{enumerate}
        
        3 tutorials i Matlab live script 
        \begin{enumerate}
            \item \textit{tutorial\_1.mlx}
            \item \textit{tutorial\_2.mlx}
            \item \textit{tutorial\_3.mlx}
        \end{enumerate}
         Total uppskattad tidsåtgång ca. 2 h
    \end{frame}
    
    \begin{frame}{Uppvärmning Matlab}
        Öppna \textit{tutorial\_1.mlx} och gör sektionerna
        \begin{itemize}
            \item \textcolor{orange}{Matlab live script}
            \item \textcolor{orange}{Tutorial 1 - Introduktion}
            \item Besvara frågorna i Studium
        \end{itemize}
        \vspace{0.5cm}

        Påbörja sedan Föreläsning 2 - Mätvärdesbehandling
        
        \vspace{0.5cm}
        \begin{center}
            \large{Lycka till!}
        \end{center}
    \end{frame}
    
\end{document}