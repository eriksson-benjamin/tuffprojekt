\documentclass[9pt]{beamer}
\usepackage{beamerpreamble}
\usepackage[swedish]{babel}
\usepackage{minted}
\usepackage{comment}
\usemintedstyle{vs}
\usepackage{xcolor}
\usepackage{tikz}
\usepackage{textgreek}
\usepackage{dirtytalk}
\usepackage{comment}

\renewcommand{\ttdefault}{cmtt}

\newcommand*\mean[1]{\bar{#1}}

\title{Datalaboration - Förberedande tutorial}
\author[benjamin.eriksson@physics.uu.se]{Benjamin Eriksson  \\ \tiny{med inspiration från} \\ \scriptsize{Slides av M. Isacson, M. Ellert, M. Olvegård, och B. Lindgren}}
\institute[Uppsala universitet]{{\small Avdelningen för tillämpad kärnfysik \\ Institutionen för fysik och astronomi} \\ \uulogo}
\date{{\small Reviderad}\\ \today}

\begin{document}
\begin{frame}{}
    Hur ser matrisen $CD$ ut efter att du utfört elementvis multiplikation?
    \begin{equation*}
        C. * D = 
        \begin{pmatrix}
        9 & 16  & 21 \\ 
        24 & 25 & 24 \\ 
        21 & 16 & 9 
        \end{pmatrix}
    \end{equation*}  
\end{frame}

\begin{frame}{}
    Vad är standardavvikelsen $u(x)$?
    \begin{itemize}
        \item Mellan 0.07 och 0.074 ger rätt svar.
    \end{itemize}
\end{frame}

\begin{frame}{}
    Vad är standardosäkerheten i medelvärdet $u(\bar{x}$?
    \begin{itemize}
        \item Mellan 0.03 och 0.033 ger rätt svar.
    \end{itemize}
\end{frame}

\begin{frame}{}
    Välj de alternativ där temperaturen och dess osäkerhet presenteras enligt tumregeln.
    \begin{itemize}
        \item $T = 15.32 \pm 0.64$
        \item $T = 15.32(64)$
    \end{itemize}
\end{frame}

\begin{frame}{}
    Vad är den propagerade osäkerheten i $F$?
    \begin{equation*}
        u_F = \sqrt{\left(u_{m_j}\frac{Gm_s}{r^2}\right)^2 + \left(u_{m_s} \frac{Gm_j}{r^2}\right)^2 + \left(u_r \frac{2Gm_jm_s}{r^3}\right)^2}
    \end{equation*}
\end{frame}

\begin{frame}{}
    Vilka alternativ är sanna för kvadratsumman S?
    \begin{itemize}
        \item $0 \leq S \leq \infty$
        \item Ju mindre S är, desto mindre är summan av avstånden från datapunkter till modellen.
    \end{itemize}
\end{frame}

\begin{frame}{}
    Antag att vi vill utföra en minstakvadratanpassning av modellen $f(x) = a_1 + a_2x_i - a_3x_i^2$ på våra uppmätta data ${x_i, y_i}$. Välj de alternativ som är sanna.
    \begin{itemize}
        \item $S = \sum\limits_i \left( y_i - a_1 - a_2x_i - a_3x_i^2\right)^2$
        \item För att hitta $a_1$, $a_2$, $a_3$ minimerar vi $S$ genom att bilda ett ekvationssystem från $\frac{\partial S}{\partial a_1} = 0$, $\frac{\partial S}{\partial a_2} = 0$ och $\frac{\partial S}{\partial a_3} = 0$.
    \end{itemize}
\end{frame}

\begin{frame}{}
    \begin{center}
        All feedback uppskattas och bidrar till att förbättra W-programmet!
    \end{center}
    
\end{frame}


\end{document}